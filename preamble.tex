
% Encoding: UTF-8

%“em”是相對长度單位,相当於所用字体中大寫“M”的寬度。
%“ex”是相對长度單位,相当於所用字体中小寫“x”的高度,此高度通常为字体尺寸的一半。

\documentclass[b5paper,twoside,zihao=-4,UTF8]{ctexbook}

\usepackage{ctex}
	\ctexset{
		section = {
			format+ = \zihao{-4} \heiti \raggedright,
			name = {,.},
			number = \arabic{section},
			beforeskip = 1.0ex plus 0.2ex minus .2ex,
			afterskip = 1.0ex plus 0.2ex minus .2ex,
			aftername = \hspace{0pt}
		},
		chapter = {
			format+ = \zihao{-3} \heiti,
			name = {,.},
			number = \chinese{chapter},
			beforeskip = 1.0ex plus 0.2ex minus .2ex,
			afterskip = 1.0ex plus 0.2ex minus .2ex,
			aftername = \hspace{0pt}
		}
	}

\usepackage{xeCJK}
	\xeCJKsetup{
		PunctStyle=kaiming
	}
	\setCJKmainfont[Path=ttf/]{TH-Sung-TP0}
	\setCJKsansfont[Path=ttf/]{TH-Sung-TP0}
	\setCJKmonofont[Path=ttf/]{TH-Sung-TP0}
%	\setCJKmainfont{SimSun}%设置正文罗马族的 CJK 字体
%	\setCJKsansfont{SimSun}%设置正文无衬线族的 CJK 字体
%	\setCJKmonofont{SimSun}%设置正文等宽族的 CJK 字体

\usepackage{fontspec}
%	\setmainfont{SimSun}
%	\setmainfont[Path=ttf/]{TH-Sy-P0}
%	\setsansfont{SimSun}

\usepackage{geometry}%頁面尺寸
	\geometry{left=2.5cm,right=2.5cm,top=2cm,bottom=2cm}

\usepackage[stable]{footmisc}

\usepackage{graphicx}
\graphicspath{{pics/}}

\newcommand\jing{\hbox{\scalebox{0.6}[1]{纟}\kern-0.3em\scalebox{0.7}[1]{巠}}}%經
\newcommand\qing{\hbox{\scalebox{0.5}[1]{车}\kern-0.15em\scalebox{0.65}[1]{巠}}}%輕
\newcommand\xu{\hbox{\scalebox{0.6}[1]{纟}\kern-0.3em\scalebox{0.7}[1]{賣}}}%續
\newcommand\rao{\hbox{\scalebox{0.6}[1]{纟}\kern-0.3em\scalebox{0.7}[1]{堯}}}%繞

\pagestyle{plain}
%\punctstyle{kaiming}
\CJKsetecglue{}%完全禁用汉字与其他内容间的空格
%\raggedright
\setlength{\parindent}{0em}%段落中第一行缩进量
\setlength{\parskip}{3ex}%段落间距
%\nofiles

\title{傷寒雜病論}

\author{
	张仲景著述\\王叔和撰次\\唐弘宇校訂
\and
	毛{\hfill}敏\\彭{\hfill}健\\楊漢鑫\\枼建橋\\鄭{\hfill}浩\\張麟玉\\趙任傑
}

\date{\today}

