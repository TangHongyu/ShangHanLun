
% frontmatter.tex
% Encoding: UTF-8

\frontmatter
\maketitle

\chapter{凡例}

\begin{itemize}
	
	\item 《傷寒雜病論匯校本》分为三个部分:第一部分,類聚方部分;第二部分,傷寒部分;第三部分,雜病部分。\\
	《類聚方》一書,是日本漢方醫學古方派代表人物吉益東洞的著作,此書分解《傷寒論》、《金匱要略》條文,以方剂为類目,匯集仲景相関論述,着意凸顯其「方証相對」之學術主張。本書的第一部分,即是以吉益東洞《類聚方》的形式,重新排列第二、第三部分的條文。方剂的先後次序基本按照《類聚方》,《類聚方》中缺失的方剂,則按照其在《傷寒雜病論》中出現的次序,排在最後。\\
	第一部分中的所有條文,与第二、第三部分的條文完全相同,但條文的註釋和校改記錄,僅在第二、第三部分有,第一部分不再出現。\\
	第二、第三部分僅有條文,无方剂,所有方剂僅出現在第一部分,所以對方剂的註釋,也僅出現在第一部分。
	
	\item 本書傷寒論部分所用的底本,是臺北故宮博物院文獻大樓藏明趙開美翻刻北宋元祐三年小字本《傷寒論》。雜病論部分所用的底本,是明吳遷钞本《金匱要略方》。本書所用的主校本有《千金翼方》、《金匱玉函經》、《脉經》,参校本有《千金要方》、《外臺祕要》、《聖惠方》、《醫心方》等。不用《康平本》、《康治本》、《桂林古本》。
	
	\item 主要参考書目:
		\begin{itemize}
			\item 《影印孫思邈本傷寒論校注考証》
			\item 《宋本傷寒論文獻史論》學苑出版社2015
			\item 《影印南朝祕本敦煌祕卷傷寒論校注考証》
			\item 《影印金匱玉函經校注考証》
			\item 《校勘元本影印明本金匱要略集》學苑出版社2015
			\item 《明洪武鈔本金匱要略方》上海科學技術文獻出版社2011
		\end{itemize}

	\item 古人校訂書籍,不修改原文,只在疑似有錯処出校注,這樣做是为了保存文獻原貌,防止由於自己理解錯誤而妄改原文。這些謹慎的作法,都是因为古代印刷技術、保存手段落後等原因,而採取的無奈之舉。而以現代的保存手段,不太可能再有古籍消失。所以,我在校訂此書時,不考慮保存原貌的問題,直接修改原文,必要時出校記,這樣做是为了使學習、閲讀更加方便。在校勘時,如能確定底本有脱漏者,即據其它校本補。如底本與校本有歧異者,即作理校,據文意來推斷正誤。如遇到義可兩存者,則在註釋中説明。
	
	\item 規範字形,將古書中的異體字、通假字,一律改为正字,對某些避諱字,也改正過來。部分字使用簡化字形。請注意,「正字」不是臺灣正體字,「簡化字」也不是現行的大陸簡體字,宋本書籍中已有使用簡化字的先例。
	
	\item 《太平聖惠方》卷二論合和曰:「古方藥味,多以銖兩,及用水,皆言升數。年代綿歷浸遠,傳寫轉見乖訛。」(待補充)。大柴胡湯、柴胡桂枝乾薑湯在《傷寒論》和《金匱要略》中重出,其中柴胡的份量,趙本《傷寒論》作「半斤」,吳本《金匱要略》作「八兩」。为{\sungii 𥁞}量統一全書單位,書中所有的「斤」單位,全部依此例轉換为「兩」單位,「一斤」合「十六兩」。
	
	\item 相比宋本系統,本書在行文上更簡潔,風格更接近《千金翼方》和《金匱玉函經》。舉一个例子,趙本第282條中的「下焦虗有寒」,《千金翼》作「下焦虗寒」,《聖惠方》作「下焦有虗寒」,它們表達的意思相同,但《千金翼》更簡潔,所以我選取了《千金翼》的説法。
	
	\item 
	
	\item 
	
	\item 

\end{itemize}

\chapter{傷寒論序}

論曰。余每覽越人入虢之診。望齐矦之色。未嘗不慨然嘆其才秀也。怪当今居世之士。曾不留神醫藥。精究方術。上以療君親之疾。下以救貧賤之厄。中以保身长全。以養其生。但競逐榮勢。企踵權豪。孜孜汲汲。惟名利是務。崇飾其末。忽弃其本。華其外而悴其内。皮之不存。毛將安附焉。卒然遭邪風之气。嬰非常之疾。患及禍至。而方震慄。降志屈節。欽望巫祝。告窮歸天。束手受敗。齎百年之壽命。持至貴之重器。委付凡醫。恣其所措。咄嗟嗚呼。厥身已斃。神明消滅。變为異物。幽潛重泉。徒为啼泣。痛夫。舉世昏迷。莫能覺悟。不惜其命。若是輕生。彼何榮勢之云哉。而進不能愛人知人。退不能愛身知已。遇災值禍。身居厄地。矇矇昧昧。蠢若游魂。哀乎。趨世之士。馳競浮華。不固根本。忘軀徇物。危若冰谷。至於是也。

余宗族素多。向餘二百。建安紀年以來。猶未十稔。其死亡者。三分有二。傷寒十居其七。感往昔之淪喪。傷橫夭之莫救。乃勤求古訓。博采乑方。\footnote{趙本此処有「撰用素問九卷八十一難陰陽大論胎臚藥錄并平脉辨証」二十三字。}为傷寒雜病論合十六卷。雖未能{\sungii 𥁞}愈諸病。庶可以見病知源。若能尋余所集。思過半矣。\footnote{趙本此段下有「夫天布五行。以運萬類。人稟五常。以有五臟。經絡府俞。陰陽會通。玄冥幽微。變化難極。自非才高識妙。豈能探其理致哉。上古有神農。黄帝。歧伯。伯高。雷公。少俞。少師。仲文。中世有长桑。扁鵲。漢有公乘陽慶及倉公。下此以往。未之聞也。觀今之醫。不念思求經旨。以演其所知。各承家技。終始順舊。省疾問病。務在口給。相對斯須。便處湯藥。按寸不及尺。握手不及足。人迎。趺陽。三部不参。動數发息。不滿五十。短期未知決診。九候曾無彷彿。明堂闕庭。{\sungii 𥁞}不見察。所謂窺管而已。夫能視死別生。實为難矣。」一段文字。}孔子云。生而知之者上。學則亞之。多聞博識。知之次也。余素尚方術。請事斯語。
	\footnote{
		關於這段序言,楊紹伊説:「仲景序中,『撰用《素問》、《九卷》、《八十一難》、《陰陽大論》、《胎臚藥錄》并《平脉辨証》』五句,与『若能寻余所集,則思過半矣』至『夫欲視死別生,実为難矣』一節,悉出其撰次。知者,以此篇序文,讀其前半,韵虽不高而清,調虽不古而雅,非駢非散,的是建安。『天布五行』与『省疾問病』二段,則笔调句律,節款声响,均屬晉音。试以《傷寒例》中詞語滴血驗之,即知其是一家骨肉。更証以《千金方》序文中引『当今居世之士,曾不留神醫藥』至『彼何榮勢之云哉』一節称『张仲景曰』,緒論中引『天布五行,以運万类』至『夫欲視死別生,実为難矣』一節,不称『张仲景曰』,即知其語非出自仲景之口。再以文律格之,『勤求古訓,博采乑方』在文法中为渾説,『撰用《素問》《九卷》』等五句,在文法中为詳舉。凡渾説者不詳舉,詳舉者不渾説。原文当是『感往昔之淪喪,傷横夭之莫救,乃勤求古訓,博采乑方,为《傷寒雜病論》,合十六卷』,此本辞自足,而体且簡。若欲詳舉,則当云『感往昔之淪喪,傷横夭之莫救,乃撰用《素問》、《九卷》、《八十一難》、《陰陽大論》《胎臚藥錄》并《平脉辨証》,为《傷寒雜病論》,合十六卷』,不当渾説后又詳舉也。且仲景为醫中之湯液家,湯液家舉书,不舉《湯液经》而舉《素問》,不数伊尹,而数岐黄,何異家乘中不系祖祢而譜東邻也?至其下之『按寸不及尺,握手不及足,人迎趺陽,三部不参』云云,殊不知三部九矦乃針灸家脉法,非湯液家脉法。針家刺在全身,勢不能不遍体考脉;湯液家重在現証,脉則但候其表里寒热,臓腑虗実,榮衛盛衰,以决其治之可汗不可汗,可下不可下而已矣。故診一部亦已可定,不必遍体摩挲。以湯液家而用針灸家骂湯液家之語骂人,仲景縱亦精於針灸脉法,何至遽憒瞀而矛盾若是。」
	}

\endinput
