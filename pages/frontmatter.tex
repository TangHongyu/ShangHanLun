
% frontmatter.tex
% Encoding: UTF-8

\frontmatter
\maketitle
\tableofcontents

\chapter{凡例}

\begin{itemize}
	
	\item 《傷寒雜病論匯校本》分为三個部分:第一部分,類聚方部分;第二部分,傷寒部分;第三部分,雜病部分。\\
	《類聚方》一書,是日本漢方醫學古方派代表人物吉益東洞的著作,此書分解《傷寒論》、《金匱要略》條文,以方剂为類目,匯集仲景相関論述,着意凸顯其「方証相對」之學術主張。本書的第一部分,即是以吉益東洞《類聚方》的形式,重新排列第二、第三部分的條文。方剂的先後次序基本按照《類聚方》,《類聚方》中缺失的方剂,則按照其在《傷寒雜病論》中出現的次序,排在最後。\\
	第一部分中的所有條文,与第二、第三部分的條文完全相同,但條文的註釋和校改記錄,僅在第二、第三部分有,第一部分不再出現。\\
	第二、第三部分僅有條文,无方剂,所有方剂僅出現在第一部分,所以對方剂的註釋,也僅出現在第一部分。
	
	\item 本書傷寒論部分所用的底本,是臺北故宮博物院文獻大樓藏明趙開美翻刻北宋元祐三年小字本《傷寒論》,雜病論部分所用的底本,是明吳遷钞本《金匱要略方》。本書所用的主校本有《千金翼方》、《金匱玉函經》、《脉經》。参校本有《千金要方》、《外臺祕要》、《聖惠方》、《醫心方》等。不用《康平本》、《康治本》、《桂林古本》。
	
	\item 主要參考書目:
		\begin{itemize}
			\item 《影印孫思邈本傷寒論校注考證》
			\item 《宋本傷寒論文獻史論》
			\item 《影印南朝祕本敦煌祕卷傷寒論校注考證》
			\item 《影印金匱玉函經校注考證》
			\item 《校勘元本影印明本金匱要略集》
			\item 《明洪武鈔本金匱要略方》
		\end{itemize}

	\item 古人校訂書籍,不修改原文,只在疑似有錯処出校注,這樣做是为了保存文獻原貌,防止由於自己理解錯誤而妄改原文。這些謹慎的作法,都是因为古代印刷技術、保存手段落後等原因,而採取的無奈之舉。在現代,保存古籍的任務有政府建立圖書館負責,而且以現代的保存手段,不太可能再有古籍消失。所以,我在校訂此書時,不考慮保存原貌的問題,直接修改原文,必要時出校記,這樣做是为了使學習、閲讀更加方便。
	
	\item 規範字形,將古書中的異體字、通假字,一律改为正字,部分字使用簡化字。請注意,「正字」不是臺灣正體字,「簡化字」也不是現行的大陸簡體字,宋本書籍中已有使用簡化字的先例。
	
	\item 《太平聖惠方》卷二論合和曰:「古方藥味,多以銖兩,及用水,皆言升數。年代綿歷浸遠,傳寫轉見乖訛。」(待補充)。大柴胡湯、柴胡桂枝乾薑湯在《傷寒論》和《金匱要略》中重出,其中柴胡的份量,趙本《傷寒論》作「半斤」,吳本《金匱要略》作「八兩」。为盡量統一全書單位,書中所有的「斤」單位,全部依此例轉換为「兩」單位,「一斤」合「十六兩」。
	
	\item 相比宋本系統,本書在行文上更簡潔,風格更接近《千金翼方》和《金匱玉函經》。舉一個例子,趙本第282條中的「下焦虗有寒」,《千金翼》作「下焦虗寒」,《聖惠方》作「下焦有虗寒」,它們表達的意思相同,但《千金翼》更簡潔,所以我選取了《千金翼》的説法。
	
	\item 
	
	\item 
	
	\item 
	
\end{itemize}























\endinput
